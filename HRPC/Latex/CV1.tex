% --------------------------------------------------------------
% This is all preamble stuff that you don't have to worry about.
% Head down to where it says "Start here"
% --------------------------------------------------------------

\documentclass[12pt]{article}

\usepackage[margin=1in]{geometry}
\usepackage{amsmath,amsthm,amssymb}
\usepackage[utf8]{inputenc}

\begin{document}

% --------------------------------------------------------------
%                         Start here
% --------------------------------------------------------------

\title{Tanque de Oxidante}
\author{José María Fernández Rodríguez \& Sebastián López Sánchez}

\maketitle

El modelo termodinámico de un motor de cohete híbrido comienza por el tanque de oxidante (Figura 1). El tanque consiste de un volumen determinado $V_{T}$ a una temperatura $T_T$, el cual se encuentra sujeto a una presión interna $P_T$, generada por la candtidad de oxidante auto-presurizante (Óxido Nitroso $N_{2}O$) en estado líquido $n_{ox,l}$ y gaseoso $n_{ox,v}$, así como por la cantidad de gas presurizante $n_{sp,v}$. \par

Figura 1: Tanque \par

A partir de la investigación, el modelo termodinámico elegido para el tanque fue el modelo ideal planteado por (Fernández), a continuación se explican las \emph{restricciones} del modelo y su derivación:

\begin{enumerate}
    \item Se asume que los contenidos del tanque se encuentran en equilibrio de fase en todo momento. El tanque se vacía de tal manera que la temperatura, $T_T$, y la presión, $P_P$, de las fases líquida y gaseosa son uniformes durante el proceso de drenado.
    \item Se desprecia el efecto de la gravedad y la aceleración en el tanque.
    \item Las paredes del tanque son adiabáticas y se encuentran en equilibrio térmico con el tanque.
    \item Se desprecia la energía potencial y cinética de los contenidos del tanque.
    \item Se asume que el comportamiento de los gases se apega a la ley de gases ideales.
    \item El gas presurizante (He) no se condensa y permanece en estado gaseoso.
    \item El flujo de oxidante hacia la cámara de combustión es debido a la caída de presión entre el tanque de oxidante y la cámara de combustión, y las pérdidas son tomadas en cuenta a través de un coeficiente de descarga.
    \item La cantidad de Helio permanece constante durante el proceso de drenado.
    \item La evaporación ocurre en la interfaz de las fases líquida y gaseosa. Debido al estado de equilibrio asumido por el modelo, no ocurre ebullición dentro del tanque.
\end{enumerate}

La variación molar del contenido de oxidante líquido en el tanque, siguiendo el principio de conservación de la masa, es descrita por la siguiente ecuación:

\begin{equation}
    \dot{n}_{ox,l} = -\dot{n}_{ox,v} - \dot{n}_{d}
    \label{descarga 1}
\end{equation}

Y la ecuación de estado estacionario para el flujo másico a través de un orificio es:

\begin{equation}
    \dot{m}_{d} = C_{d} A_{inj} \sqrt{2\rho_{ox,l}(P_{T} - P_{c})}
\end{equation}

En su forma molar:

\begin{equation}
    \dot{n}_{d} = C_{d} A_{inj} \sqrt{ \frac{ 2\rho_{ox,l}(P_{T} - P_{c}) }{ (MW)_{ox} \overline{V}_{ox,l} } }
    \label{descarga 2}
\end{equation}

Donde $C_{d}$ es el coeficiente de descarga, $A_{inj}$ es el área de los inyectores, $(MW)_{ox}$ es el peso molecular del oxidante, y $\overline{V}_{ox,l}$ es el volumen molar del óxido nitroso. Despejando la ecuación \ref{descarga 1} para $\dot{n}_{ox,l}$ y $\dot{n}_{ox,v}$ obtenemos la primera ecuación necesaria para resolver el sistema:

\begin{equation}
    \dot{n}_{ox,v} + \dot{n}_{ox,l} = -C_{d}N_{inj}A_{inj}\sqrt{\frac{2(P_{T}-P_{c})}{(MW)_{ox}\overline{V}_{ox,l}}}
    \label{ox 1}
\end{equation}

Para la siguiente ecuación, se toma ventaja de la restricción de que el volumen del tanque permanece constante, la ley de presiones parciales de Raoult, y la ecuación de gases ideales.

La restrición de volumen constante puede ser descrita por la siguiente ecuación:

\begin{equation}
    V_{T} = V_{g} + V_{l} = V_{sp,v} + V_{ox,v} + n_{ox,l} \overline{V}_{ox,l}
    \label{volumen}
\end{equation}

La ecuación de equilibrio de la ley de presiones parciales de Raoult relaciona la presión total de los gases con la presión soportada por cada uno de los gases, y en el caso actual es la siguiente:

\begin{equation}
    P_{T} \left( \frac{n_{ox,v}}{n_{sp,v}+n_{ox,v}} \right) = P^{*}_{ox,v}
    \label{raoult}
\end{equation}

Por último tenemos la ecuación de gases ideales, expresada con las variables del modelo, y tomando en cuenta la restricción de la ecuación \ref{volumen} queda de la siguiente manera:

\begin{equation}
    P_{T}(V_{T} - n_{ox,l}  \bar{V}_{ox,l})=(n_{sp,v}+n_{ox,v}) R_{u} T_{T}
    \label{ideales}
\end{equation}

Al combinar las ecuaciones \ref{raoult}, y \ref{ideales}, eliminando la variable $P_{T}$

\begin{equation}
    P^{*}_{ox,v} (V_{T} - n_{ox,l} \bar{V}_{ox,l})  = n_{ox,v} R_{u} T_{T}
\end{equation}

Derivando con respecto al tiempo, y usando la regla de la cadena

\begin{equation}
    (V_{T} - n_{ox,l}\overline{V}_{ox,l}) \frac{P^{*}_{ox,v}}{dT} \dot{T}_{T} - \dot{n}_{ox,l} \overline{V}_{ox,l} P^{*}_{ox,v} = R_{u} \left( T_{T} \dot{n}_{ox,v} + \dot{T}_{T} n_{ox,v} \right)
    \label{ox 2}
\end{equation}

La ecuación \ref{ox 2}, donde las incógnitas son $\dot{n}_{ox,l}$, $\dot{n}_{ox,v}$ y $\dot{T}_{T}$, es la segunda ecuación necesaria para modelar el sistema.

Ahora, haciendo un balance de energía en el volumen de control de la Figura REFERENCIA, según la primera ley de la termodinámica para sistemas abiertos $\dot{U} = \dot{Q} + \dot{W} + \dot{m}h$; y tomando en cuenta que, de acuerdo con las restricciones, el proceso es adiabático $\dot{Q} = 0$ y que no existe trabajo de frontera $\dot{W} = 0$, el balance de energía queda de la manera siguiente:

\begin{equation}
  \frac{d}{dt}\left( m_{T} u_{T} + n_{ox,l} \overline{U}_{ox,l} + n_{ox,v} \overline{U}_{ox,v} + n_{sp,v} \overline{U}_{sp,v} \right) = -\frac{d}{dt}(n_{ox,l} + n_{ox,v})\overline{H}_{ox,l}
\end{equation}

Donde $m_{t}$ es la masa del tanque y $u_{T}$ es la energía específica del tanque. Usando la regla de la cadena con los términos de la energía interna:

\begin{equation}
  \begin{split}
    m_{T} \dot{u}_{T} + n_{ox,l} \dot{\overline{U}}_{ox,l} + n_{ox,v} \dot{\overline{U}}_{ox,v} + n_{sp,v} \dot{\overline{U}}_{sp,v} = \\
    \dot{n}_{ox,l}(\overline{H}_{ox,l} - \overline{U}_{ox,l}) + \dot{n}_{ox,v}(\overline{H}_{ox,l} - \overline{U}_{ox,v})
  \end{split}
\end{equation}

A continuación haciendo uso de lo siguiente: la entalpía de vaporización está relacionada a la entalía líquida por ${H}_{ox,v} - {H}_{ox,l} = \delta {H}_{ox,v}$; la definición de la entalpía para los gases $H = U + PV$; la ecuación de gases ideales $P\overline{V} = RT$; la definición del calor específico a volumen constante $d\overline{U} = C_{V}dT$; y que $C_{V} \approx C_{P}$ para materiales sólidos; simplificamos la ecuación:

\begin{equation}
  \begin{split}
    ( m_{T} c_{P_{T}} + n_{ox,l} \overline{C}_{V_{ox,l}} + n_{ox,v} \overline{C}_{V_{ox,v}} + n_{sp,v} \overline{C}_{V_{sp,v}} ) \frac{dT_{T}}{dt} = \\
    \dot{n}_{ox,l}(P\overline{V}_{ox,l}) + \dot{n}_{ox,v}(R_{u} T_{T} - \Delta {H}_{ox,v})
    \label{ox 3}
  \end{split}
\end{equation}

La ecuación \ref{ox 3}, donde las incógnitas son $\dot{n}_{ox,l}$, $\dot{n}_{ox,v}$ y $\dot{T}_{T}$, es la tercera y última ecuación necesaria para resolver el sistema de tanque de oxidante descrito.

% --------------------------------------------------------------
%     You don't have to mess with anything below this line.
% --------------------------------------------------------------

\end{document}
